\begin{sang}{Fulbert og Beatrice}{Jens Louis Petersen}%
\begin{multicols}{2}\multicolinit
\begin{vers}
I frankens rige, hvor floder rinde
\verseend
som sølverstrømme i lune dal,
\verseend
lå ridderborgen på bjergets tinde
\verseend
med slanke tårne og gylden sal.
\verseend
Og det var sommer med blomsterbrise
\verseend
og suk af elskov i urtegård.
\verseend
Og det var Fulbert og Beatrice,
\verseend
og Beatrice var sytten år.
\end{vers}
\begin{vers}
De havde leget som børn på borgen,
\verseend
mens Fulbert endnu var gangerpilt.
\verseend
Men langvejs drog han en årle morgen,
\verseend
mod Saracenen han higed' vildt.
\verseend
Han spidded' tyrker som pattegrise,
\verseend
et tusind stykker blev lagt på bår,
\verseend
for Fulbert kæmped' for Beatrice,
\verseend
og Beatrice var sytten år.
\end{vers}
\begin{vers}
Med gluttens farver på sølversaddel
\verseend
han havde stridt ved Jerusalem.
\verseend
Han kæmped kækt uden frygt og dadel
\verseend
og gik til fods hele vejen hjem.
\verseend
Nu sad han atter på bænkens flise
\verseend
og viste stolt sine heltesår,
\verseend
som ganske henrykked' Beatrice
\verseend
for Beatrice var sytten år.
\end{vers}
\begin{vers}
En kappe prydet med små opaler
\verseend
og smagfuldt ternet med tyrkens blod,
\verseend
en ring af guld og et par sandaler
\verseend
den ridder lagde for pigens fod.
\verseend
Og da hun øjnede hans caprise
\verseend
blev hjertet mygt i den væne mår.
\verseend
Af lykke dånede Beatrice,
\verseend
for Beatrice var sytten år.
\end{vers}
\begin{vers}
Da banked' blodet i heltens tinding,
\verseend
thi ingen helte er gjort af træ.
\verseend
Til trods for plastre og knæforbinding
\verseend
sank ridder Fulbert med stil på knæ.
\verseend
Han kvad: ''Skønjomfru, oh skænk mig lise,
\verseend
thi du alene mit hjerte rår!''
\verseend
''Min helt, min ridder,'' kvad Beatrice,
\verseend
for Beatrice var sytten år.
\end{vers}
\begin{vers}
Og der blev bryllup i højen sale
\verseend
med guldpokaler og troubadour,
\verseend
og under sange og djærven tale
\verseend
blev Fulbert ført til sin jomfrus bur.
\verseend
Og følget hvisked' om øm kurtice
\verseend
og skæmtsom puslen blandt dun og vår.
\verseend
For det var Fulbert og Beatrice,
\verseend
og Beatrice var sytten år.
\end{vers}
\begin{vers}
Men ridder Fulbert den samme aften
\verseend
af borgens sale blev båren død.
\verseend
Den megen krig havde tær't på kraften,
\verseend
og sejrens palmer den sidste brød.
\verseend
Oh bejler, lær da af denne vise:
\verseend
Ød ej din kraft under krigens kår.
\verseend
Nej, spar potensen til Beatrice,
\verseend
når Beatrice er sytten år.
\end{vers}
\end{multicols}
\end{sang}
